\chapter*{\centering{\large{ABSTRAK}}}

\begin{spacing}{1}
\textbf{AKBAR MAULANA ALFATIH}. Ekspansi Aplikasi Aqua Breeding Dengan Penambahan Fitur Inventarisasi Untuk Penentuan Harga Dasar Produk Perikanan Berbasis Android. Skripsi. Fakultas Matematika dan Ilmu Pengetahuan Alam, Universitas Negeri Jakarta. 2023. Di bawah bimbingan Muhammad Eka Suryana, M.Kom dan Med Irzal, M.Kom.
\newline
\newline
Budidaya perikanan air tawar merupakan salah satu sumber perikanan yang ada di Indonesia. Dalam berbudidaya, tentunya penting untuk mencatat indikator-indikator inventaris budidaya ikan seperti pakan ikan, suplemen ikan, aset kolam, listrik pada kolam, serta benih ikan yang berguna untuk menentukan harga jual ikan. Penelitian ini bertujuan untuk  memperluas aplikasi Aqua Breeding dengan menambahkan fitur inventarisasi yang dapat digunakan untuk mencatat penggunaan inventaris serta menentukan harga jual minimum ikan yang jujur. Data pada penelitian ini diambil dari hasil diskusi bersama pembudidaya ikan air tawar JFT (J Farm Technology) dan studi literatur dengan membaca jurnal-jurnal yang terkait dengan topik penelitian. Diskusi tersebut menghasilkan suatu user requirement yang menjadi pedoman dalam membuat web service pada backend serta penerapannya pada frontend mobile. Metode pengembangan sistem ini menggunakan metode Scrum dengan jumlah Sprint sebanyak lima Sprint serta teknologi yang digunakan adalah Flask dengan bahasa Python pada backend dan Flutter dengan bahasa Dart pada frontend. Hasil akhir dari penelitian ini adalah web service berupa REST API berserta dokumentasinya dan juga penerapannya pada aplikasi berbasis Android yang telah diuji dengan metode pengujian unit testing dan \textit{User Acceptance Test} (UAT).
\newline
\newline
\noindent \textbf{Kata kunci}: \textit{sistem inventarisasi, aplikasi mobile, transaksi ikan, budidaya perikanan modern, scrum}
\end{spacing}