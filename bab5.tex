%!TEX root = ./template-skripsi.tex
%-------------------------------------------------------------------------------
%                            	BAB IV
%               		KESIMPULAN DAN SARAN
%-------------------------------------------------------------------------------

\chapter{KESIMPULAN DAN SARAN}

\section{Kesimpulan}
Berdasarkan hasil implementasi dan pengujian fitur aplikasi yang telah dirancang, maka diperoleh kesimpulan sebagai berikut:

\begin{enumerate}
	\item Terbentuknya aplikasi Aqua Breeding versi kedua dengan fitur sistem inventarisasi serta penentuan harga jual minimum ikan. Adapun perancangan aplikasi ini dilakukan dengan metode Scrum dimulai dari tahap penyusunan Product Backlog, Sprint Backlog, dan dikerjakan dalam lima Sprint.
	\item Berdasarkan hasil pengujian, seluruh skenario pada unit testing berjalan dengan baik.
\end{enumerate}

\section{Saran}
Adapun saran untuk penelitian selanjutnya adalah:
\begin{enumerate} 
	\item Berdasarkan diskusi dengan owner, pada versi selanjutnya adalah penambahan fitur transaksi antar pembudidaya ikan yang berguna untuk penjualan ikan agar harga ikan hasil panen dapat digunakan.
\end{enumerate}
% Baris ini digunakan untuk membantu dalam melakukan sitasi
% Karena diapit dengan comment, maka baris ini akan diabaikan
% oleh compiler LaTeX.
\begin{comment}
\bibliography{daftar-pustaka}
\end{comment}