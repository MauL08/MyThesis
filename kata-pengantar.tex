\chapter*{\centering{\large{KATA PENGANTAR}}}

Puji syukur penulis panjatkan ke hadirat Allah SWT, karena dengan rahmat dan karunia-Nya, penulis dapat menyelesaikan proposal skripsi yang berjudul \textbf{"Perancangan Arsitektur \textit{Search Engine} dengan Mengintegrasikan \textit{Web Crawler}, Algoritma \textit{Page Ranking}, dan \textit{Document Ranking}"}.

Keberhasilan dalam menyusun proposal skripsi ini tidak lepas dari bantuan berbagai pihak yang mana dengan tulus dan ikhlas memberikan masukan guna sempurnanya proposal skripsi ini. Oleh karena itu dalam kesempatan ini, dengan kerendahan hati penulis mengucapkan banyak terima kasih kepada:

\begin{enumerate}

	\item{Yth. Para petinggi di lingkungan FMIPA Universitas Negeri Jakarta.}
	\item{Yth. Ibu Ir. Fariani Hermin Indiyah, M.T selaku Koordinator Program Studi Ilmu Komputer.}
	\item{Yth. Bapak Muhammad Eka Suryana, M.Kom selaku Dosen Pembimbing I yang telah membimbing, mengarahkan, serta memberikan saran dan koreksi terhadap proposal skripsi ini.}
	\item{Yth. Bapak Drs. Mulyono, M.Kom selaku Dosen Pembimbing II yang telah membimbing, mengarahkan, serta memberikan saran dan koreksi terhadap proposal skripsi ini.}
	\item{Ayah dan Ibu penulis yang selama ini telah mendukung dan membantu menyelesaikan proposal skripsi ini.}
	\item{Teman-teman Program Studi Ilmu Komputer 2018 yang telah mendukung dan membantu proposal skripsi ini.}
	
\end{enumerate}

Penulis menyadari bahwa penyusunan proposal skripsi ini masih jauh dari sempurna karena keterbatasan ilmu dan pengalaman yang dimiliki. Oleh karenanya, kritik dan saran yang bersifat membangun akan penulis terima dengan senang hati. Akhir kata, penulis berharap tugas akhir ini bermanfaat bagi semua pihak khususnya penulis sendiri. Semoga Allah SWT senantiasa membalas kebaikan semua pihak yang telah membantu penulis dalam menyelesaikan proposal skripsi ini.

\vspace{4cm}

\begin{tabular}{p{7.5cm}c}
	&Jakarta, 6 Juni 2022\\
	&\\
	&\\
	&\\
	&Lazuardy Khatulistiwa
\end{tabular}