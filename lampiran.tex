\addcontentsline{toc}{chapter}{LAMPIRAN}
\appendix
\chapter {Transkrip Percakapan}
\begin{flushleft}
Hari: Rabu
\linebreak
Tanggal: 15 Maret 2023
\linebreak
PL: Penulis
\linebreak
KL: Klien (Pemilik Farm)
\linebreak
\linebreak
PL: Sistem apa yang akan di buat?
\linebreak
KL: Kita akan membuat sistem inventaris pada budidaya perikanan untuk penentuan harga jual minimum ikan
\linebreak
PL: Apa saja requirement dan fitur yang dibutuhkan oleh sistem ini?
\linebreak
KL: Fitur utama yang harus ada adalah sistem inventaris benih, inventaris pakan, inventaris suplemen, inventaris listrik, dan inventaris aset
\linebreak
PL: Bagaimana cara menentukan harga jual minimum ikan dari sistem inventaris tersebut?
\linebreak
KL: Dengan cara menghitung pengeluaran benih, pakan, suplemen, aset serta pengeluaran listrik per kolam aktif selama musim budidaya berjalan. Kemudian, pengeluaran tersebut akan dibagi dengan total ikan ketika masa panen tiba.
\linebreak
PL: Apakah dengan masuknya inventaris aset akan mempengaruhi nilai jual ikan per musim panen?
\linebreak
KL: Dengan masuknya harga aset kedalam perhitungan, harga ikan otomatis akan menjadi tinggi. Untuk itu, harga aset dibagi dengan 5 tahun masa pemakaian karena masa aset dalam inventaris bisa berlangsung lama. Dalam kasus ini diubah menjadi 60 bulan karena musim panen bisa dilakukan tiap bulan.
\linebreak
PL: Dimulai dari manakah pengerjaan fitur-fitur tersebut?
\linebreak
KL: Dimulai dari inventaris benih
\end{flushleft}