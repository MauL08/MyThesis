%!TEX root = ./template-skripsi.tex
%-------------------------------------------------------------------------------
%                            BAB III
%               			PEMBAHASAN
%-------------------------------------------------------------------------------

\chapter{METODOLOGI PENELITIAN}

\section{Keterhubungan Penelitian}

Pada diagram diatas, terdapat diagram alur penelitian dari aplikasi Aquaculture Tech. Diagram tersebut menjelaskan bahwa penelitian yang dilakukan oleh Fadhil \citep{fadhil2022}, 

\section{Metode Penentuan Nilai Jual}

Dalam menentukan nilai jual, dapat digunakan rumus dibawah ini.

\section{Analisa Arsitektur Fitur}

Pada penelitian aplikasi yang sudah dikembangkan sebelumnya, terdapat use case yang menjelaskan konsep dari aplikasi yang ada pada gambar dibawah ini.


\section{Analisa Pengembangan Fitur}

Pada analisa arsitektur fitur yang ada di aplikasi sebelumnya, dapat dilengkapi dengan fitur inventaris yang akan dilakukan pada penelitian ini. Fitur tersebut memiliki use case diagram seperti berikut.

% \section{Stories}

%     \subsection{Pencatatan Inventaris}

%     \begin{enumerate}
%         \item Bahan baku
%         \begin{enumerate}
%             \item Pakan
%             \item Bahan organik
%         \end{enumerate}
%         \item Listrik
%         \item Benih
%     \end{enumerate}

%     \subsection{Depresiasi Aset}

%     \begin{enumerate}
%         \item Kadaluarsa bahan baku
%         \item Penurunan kualitas
%     \end{enumerate}