\chapter*{\centering{\large{ABSTRACT}}}

\begin{spacing}{1}
\textbf{AKBAR MAULANA ALFATIH}. Aqua Breeding Application Expansion With Addition of Inventory Feature for Determining Base Price of Fishery Products Android Based. Thesis. Faculty of Mathematics and Natural Sciences, State University of Jakarta. 2023. Under the guidance of Muhammad Eka Suryana, M.Cs and Med Irzal, M.Cs.
\newline
\newline
\textit{Freshwater aquaculture is one of the sources of fisheries in Indonesia. In cultivating, of course it is important to record fish farming inventory indicators such as fish feed, fish supplements, pond assets, electricity in ponds, and fish seeds which are useful for determining the selling price of fish. This study aims to expand the Aqua Breeding application by adding an inventory feature that can be used to record inventory usage and determine an honest minimum selling price for fish. The data in this study were taken from the results of discussions with freshwater fish cultivators JFT (J Farm Technology) and literature studies by reading journals related to the research topic. The discussion resulted in a user requirement that became a guide in creating web services on the backend and its application on the mobile frontend. The system development method uses the Scrum method and the technology used is Flask with Python on the backend and Flutter with Dart on the frontend. The end result of this research is a web service in the form of a REST API along with its documentation and also its application to android-based applications.}
\newline
\newline
\noindent \textbf{Kata kunci}: \textit{inventory system, mobile application, fish transaction, modern aquaculture, scrum}
\end{spacing}