 %!TEX root = ./template-skripsi.tex
%-------------------------------------------------------------------------------
%                            BAB II
%               KAJIAN TEORI
%-------------------------------------------------------------------------------

\chapter{KAJIAN PUSTAKA} 

\section{Pengertian Persediaan dan Manajemen Persediaan}

Persediaan adalah sebuah stok barang yang dimiliki oleh sebuah perusahaan. Persediaan dapat berupa bahan mentah, bahan baku, barang jadi, barang dalam proses, hingga bahan pembantu. Persediaan atau stok barang merupakan aset perusahaan yang berharga, karena hal ini berkaitan erat dengan proses produksi. Persediaan yang tidak terstruktur akan membuat perusahaan merugi, sehingga penting untuk menerapkan manajemen persediaan dalam sebuah bisnis atau usaha.

Manajemen persediaan adalah sebuah cara untuk melakukan pengawasan, kontrol, pengelolaan terhadap persediaan atau stok barang yang dimiliki oleh sebuah perusahaan. Segala bentuk kegiatan atau aktivitas yang berkaitan dengan memperoleh, menyimpan, hingga menggunakan persediaan merupakan bagian dari manajemen persediaan.

Manajemen persediaan memiliki beberapa fungsi, yaitu:
\begin{enumerate}
	\item Mencegah terjadinya kekurangan persediaan.
	\item Mencegah barang dari supplier tidak sesuai kebutuhan.
	\item Memastikan proses produksi berjalan dengan lancar.
	\item Mengantisipasi permintaan yang mendadak.
	\item Menyesuaikan pembelian dengan jadwal produksi.
\end{enumerate}

Selain beberapa fungsi yang sudah disebutkan diatas, Manajemen persediaan juga memiliki tujuan. Setiap manajemen yang dilakukan pasti memiliki tujuan yang ingin dicapai, beberapa tujuan dari Manajemen persediaan adalah sebagai berikut.
\begin{enumerate}
	\item Mengantisipasi kenaikan harga dari bahan baku.
	\item Memastikan stok atau persediaan barang selalu tersedia.
	\item Mengurangi resiko bahan baku yang datang terlambat.
	\item Menjaga jumlah persediaan yang ada di pasaran tetap stabil.
	\item Mengantisipasi kemungkinan adanya perubahan, baik dari segi penawaran maupun permintaan.
\end{enumerate}

\section{Jenis-jenis Manajemen Persediaan}

Manajemen persediaan dibagi menjadi beberapa jenis, diantaranya:

\begin{enumerate}
	\item Bahan Mentah
	
	Bahan mentah atau biasa yang disebut dengan bahan baku, merupakan bahan utama atau dasar dari dibuatnya suatu produk. Tanpa adanya bahan baku, maka produk yang dijual tidak akan bisa untuk diproduksi.

	Bahan mentah memiliki peran yang paling penting dalam memproduksi suatu barang/produk. Untuk itu, manajemen persediaan diperlukan dalam mengelola bahan baku agar bahan baku yang diperlukan selalu tersedia dan siap untuk diproses.
	
	\item Barang Setengah Jadi
	
	Barang setengah jadi atau bisa disebut sebagai barang dalam proses merupakan barang yang belum sepenuhnya bisa digunakan, sehingga perlu untuk diproses lebih lanjut untuk menjadi barang jadi, yang nantinya siap untuk digunakan.

	Manajemen persediaan berguna untuk menghitung besar serta banyaknya barang setengah jadi tersebut untuk memenuhi kebutuhan pasar.

	\item Barang Jadi
	
	Barang jadi merupakan bahan mentah yang diproses menjadi barang setengah jadi, lalu diproses kembali sehingga menjadi barang jadi. Barang jadi bisa dibilang barang yang sudah siap untuk dijual kepada konsumen.

	Manajemen persediaan berguna untuk mengatur pengiriman produk-produk tersebut ke pasar sehingga keadaan produk di pasar tetap stabil.
\end{enumerate}

\section{Biaya Persediaan}

Penetapan biaya persediaan atau evaluasi persediaan memungkinkan perusahaan untuk memberikan nilai moneter untuk barang-barang dalam persediaan mereka. Inventaris perusahaan seringkali merupakan aset terbesarnya dan pengukuran yang tepat untuk memastikan keakuratan laporan keuangan.

Untuk menentukan biaya persediaan, diperlukan lima langkah-langkah sebagai berikut.

\begin{enumerate}
	\item Menentukan periode waktu tertentu dimana Anda perlu menemukan nilai inventaris anda.
	\item Memastikan stok atau persediaan barang selalu tersedia.
	\item Mengurangi resiko bahan baku yang datang terlambat.
	\item Menjaga jumlah persediaan yang ada di pasaran tetap stabil.
	\item Mengantisipasi kemungkinan adanya perubahan, baik dari segi penawaran maupun permintaan.
\end{enumerate}

Dalam bisnis modern, terdapat tiga metode yang digunakan dalam menghitung biaya persediaan, yaitu:

\begin{enumerate}
	\item First In, First Out (FIFO)
	
	\textit{First-in, first-out} atau FIFO adalah metode dimana aset yang diproduksi dan diperoleh terlebih dahulu juga dijual atau digunakan terlebih dahulu. Saat menggunakan FIFO sebagai metode pilihan, gunakan perhitungan ini:
	
	\textbf{HPP = biaya persediaan terlama $\times$ jumlah persediaan yang terjual}

	Contoh FIFO:

	Gunakan data tabel dibawah ini untuk menentukan HPP menggunakan metode FIFO.

	\begin{table}[h!]
		\begin{center}
		  \caption{Contoh data dalam persediaan barang}
		  \label{tab:table1}
		  \begin{tabular}{l|c|r} % <-- Alignments: 1st column left, 2nd middle and 3rd right, with vertical lines in between
			\textbf{Tanggal} & \textbf{Deskripsi} & \textbf{Unit dan Biaya} \\
			\hline
			1 Maret & Inventaris awal & 65 unit, \${15} per unit\\
			5 Maret & Beli &  130 unit, \${10} per unit\\
			10 Maret & Jual &  95 unit, \${20} per unit\\
			12 Maret & Beli &  35 unit, \${15} per unit\\
			15 Maret & Beli &  75 unit, \${20} per unit\\
			21 Maret & Jual &  115 unit, \${30} per unit\\
			28 Maret & Jual &  60 unit, \${35} per unit\\
		  \end{tabular}
		\end{center}
	  \end{table}

	  \begin{enumerate}
		\item \textbf{Tentukan jumlah unit yang tersedia untuk dijual}
		
		Tentukan jumlah unit yang tersedia untuk dijual dengan menjumlahkan semua unit beli dengan inventaris awal.

		\textbf{Unit yang tersedia untuk dijual = 65 + 130 + 35 + 75}

		\textbf{Unit yang tersedia untuk dijual = 305}

		Terdapat 305 unit yang tersedia untuk dijual.

		\item \textbf{Tentukan jumlah unit yang dijual}
		
		Tentukan jumlah unit yang dijual dengan menjumlahkan semua unit yang dijual.

		\textbf{Unit dijual = 95 + 115 + 60}

		\textbf{Unit dijual = 270}

		Ada 270 unit yang dijual.

		\item \textbf{Tentukan nilai inventaris akhir}
		
		Tentukan nilai inventaris akhir dengan cara mengurangi jumlah unit yang tersedia untuk dijual dengan jumlah unit yang dijual.

		\textbf{Inventaris akhir = Jumlah unit yang tersedia untuk dijual - jumlah unit yang dijual}

		\textbf{Inventaris akhir = 305 - 270}

		\textbf{Inventaris akhir = 35}

		Ada 35 unit yang tersisa pada inventaris akhir.

		\item \textbf{Menentukan HPP menggunakan rumus FIFO}
		
		Biaya persediaan terlama dapat ditentukan dengan nilai inventaris awal. Nilai tersebut didapat dengan mengalikan persediaan awal dan harga per unitnya. 

		\textbf{HPP = biaya persediaan terlama $\times$ jumlah persediaan yang dijual}

		\textbf{HPP = (65 $\times$ 15) $\times$ 270}

		\textbf{HPP = 263,250}

		Biaya barang yang dijual adalah \${263,250}.

	  \end{enumerate}

	\item Last In, First Out (LIFO)
	
	\textit{Last-in, first-out} atau LIFO adalah metode yang mencatat barang-barang yang baru saja diproduksi sebagai barang yang terjual lebih dulu. Saat menggunakan LIFO sebagai metode pilihan, gunakan perhitungan ini:

	\textbf{HPP = biaya persediaan terakhir $\times$ jumlah persediaan yang terjual}

	Contoh LIFO:

	Gunakan data tabel dibawah ini untuk menentukan HPP menggunakan metode LIFO.

	\begin{table}[h!]
		\begin{center}
		  \caption{Contoh data dalam persediaan barang}
		  \label{tab:table2}
		  \begin{tabular}{l|c|r} % <-- Alignments: 1st column left, 2nd middle and 3rd right, with vertical lines in between
			\textbf{Tanggal} & \textbf{Deskripsi} & \textbf{Unit dan Biaya} \\
			\hline
			3 April & Inventaris awal & 70 unit, \${15} per unit\\
			6 April & Beli &  125 unit, \${10} per unit\\
			10 April & Beli &  90 unit, \${20} per unit\\
			13 April & Jual &  40 unit, \${15} per unit\\
			16 April & Beli &  50 unit, \${15} per unit\\
			23 April & Jual &  100 unit, \${20} per unit\\
			29 April & Jual &  70 unit, \${20} per unit\\
		  \end{tabular}
		\end{center}
	  \end{table}

	\begin{enumerate}
		\item \textbf{Menentukan biaya persediaan terbaru}
		
		Untuk mencari biaya persediaan terbaru dapat dilakukan dengan menggunakan data pada inventaris awal.

		\textbf{Biaya persediaan terbaru = 70 $\times$ 15}

		\textbf{Biaya persediaan terbaru = 1,050}

		Biaya persediaan terbaru adalah \${1,050}.

		\item \textbf{Temukan jumlah unit yang dijual}
		
		Temukan jumlah unit yang terjual dengan menambahkan semua unit yang dijual.

		\textbf{Jumlah unit yang dijual = 40 + 100 + 70}

		\textbf{Jumlah unit yang dijual = 210}

		Ada 210 unit yang dijual.

		\item Gunakan rumus LIFO
		
		Gunakan biaya persediaan terbaru dan total unit yang dijual untuk menentukan nilai HPP-nya.

		\textbf{HPP = biaya persediaan terbaru $\times$ jumlah persediaan yang dijual}

		\textbf{HPP = 1,050 $\times$ 210}

		\textbf{HPP = 220,500}

		Biaya barang yang dijual adalah \${220,500}.

	\end{enumerate}

	\item Rata-rata tertimbang
	
	Rata-rata tertimbang atau biaya rata-rata tertimbang yang biasa dikenal sebagai \textit{Weighted Average Cost} (WAC) adalah metode yang menentukan jumlah masuk ke HPP dan persediaan melalui penggunaan rata-rata tertimbang. Saat menggunakan WAC, gunakan perhitungan ini:

	\textbf{WAC per unit = harga pokok barang yang tersedia / unit yang tersedia}

	Contoh WAC:

	Gunakan data tabel dibawah ini untuk menentukan rata-rata tertimbang dengan rumus WAC.

	\begin{table}[h!]
		\begin{center}
		  \caption{Contoh data dalam persediaan barang}
		  \label{tab:table3}
		  \begin{tabular}{l|c|r} % <-- Alignments: 1st column left, 2nd middle and 3rd right, with vertical lines in between
			\textbf{Tanggal} & \textbf{Deskripsi} & \textbf{Unit dan Biaya} \\
			\hline
			3 Mei & Inventaris awal & 30 unit, \${15} per unit\\
			6 Mei & Beli &  50 unit, \${15} per unit\\
			11 Mei & Jual &  25 unit, \${25} per unit\\
			15 Mei & Jual &  30 unit, \${10} per unit\\
			18 Mei & Beli &  15 unit, \${25} per unit\\
			22 Mei & Beli &  25 unit, \${35} per unit\\
			25 Mei & Jual &  35 unit, \${30} per unit\\
		  \end{tabular}
		\end{center}
	  \end{table}

	\begin{enumerate}
		\item Tentukan biaya setiap penjualan
		
		Menentukan biaya setiap penjualan dengan menghitung harga dari masing-masing semua unit yang dijual.

		\textbf{Penjualan tanggal 11 Mei} = 25 $\times$ 25 = \textbf{\${625}}

		\textbf{Penjualan tanggal 15 Mei} = 30 $\times$ 10 = \textbf{\${300}}

		\textbf{Penjualan tanggal 25 Mei} = 35 $\times$ 30 = \textbf{\${1,050}}

		\item Menjumlahkan semua biaya setiap penjualan
		
		Menggabungkan semua biaya setiap penjualan dari unit yang dijual.

		\textbf{Total biaya barang yang dijual = 625 + 300 + 1,050}

		\textbf{Total biaya barang yang dijual = \${1,975}}

		\item Temukan unit yang tersedia untuk dijual
		
		Totalkan semua unit yang tersedia untuk dijual dengan inventaris awal.

		\textbf{Total unit yang tersedia untuk dijual = 30 + 50 + 15 + 25}

		\textbf{Total unit yang tersedia untuk dijual = 120 unit}

		\item Gunakan rumus rata-rata tertimbang
		
		\textbf{WAC per unit = total biaya unit yang dijual / total unit yang dijual}
		
		\textbf{WAC per unit = 1,975 / 120}
		
		\textbf{WAC per unit = \${16.49}}

		Dari perhitungan tersebut, didapat biaya rata-rata tertimbang atau WAC-nya adalah \${16.49}
	\end{enumerate}
\end{enumerate}

\section{Perputaran Persediaan}

Perputaran Persediaan mengukur keefektifan pengelolaan persediaan dan menggambarkan efisiensi perusahaan dalam mengelola persediaannya. Semakin tinggi tingkat perputaran persediaannya maka akan semakin efektif pengelolaan persediaannya. Berikut adalah rumus untuk menentukan perputaran persediaan.

\textbf{Perputaran Persediaan = Harga Pokok Penjualan $\div$ Rata-rata Persediaan}

\section{EOQ}

\textit{Economic order quantity} (EOQ) merupakan jumlah persediaan yang digunakan untuk meminimalkan jumlah dan biaya pemesanan yang terkait dengan bahan baku atau persediaan barang dagangan. Intinya, EOQ merupakan \textit{set point} yang dibuat dan digunakan untuk menjadi acuan dalam membantu perusahaan meminimalkan total biaya persediaan.

Dua faktor penting yang menjadi penentu dalam menentukan \textit{economic order quantity} (EOQ) adalah biaya pemesanan dan biaya penyimpanan.

\begin{enumerate}
	\item Biaya pemesanan
	
	Biaya pemesanan merupakan biaya yang dikeluarkan setiap pesanan. Contoh hal yang termasuk biaya pemesanan adalah biaya pengiriman, biaya pemrosesan pembayaran, dan lain-lain.

	\item Biaya persediaan
	
	Biaya persediaan merupakan biaya yang dikeluarkan untuk menyimpan persediaan di toko atau gudang. Contoh hal yang termasuk dalam biaya penyimpaan adalah biaya sewa ruang penyimpanan, pajak properti, dan lain-lain.
\end{enumerate}

Formula atau rumus yang digunakan untuk menentukan EOQ adalah:

\[EOQ=\sqrt{\frac{2 \times D \times Co}{Ch}}\]

\begin{itemize}
	\item D = \textit{Demand per year} (Kebutuhan per tahun)
	\item Co = \textit{Cost per order} (Biaya per pesanan)
	\item Ch = \textit{Cost of holding per unit of inventory} (Biaya persediaan per unit)
\end{itemize}

\textbf{Contoh Kasus}

Sebuah material DX digunakan rutin setiap tahunnya. Data kebutuhan per tahun, biaya pemesanan, dan biaya persediaan per unit adalah sebagai berikut.

\begin{itemize}
	\item Kebutuhan tahunan = 2,400 unit
	\item Biaya per pesanan = \${10} per pesanan
	\item Biaya persediaan per unit = \${0.30} per unit
\end{itemize}

Diketahui:
\begin{itemize}
	\item D = \textit{Demand per year} (Kebutuhan per tahun) -> 2,400
	\item Co = \textit{Cost per order} (Biaya per pesanan) -> \${10}
	\item Ch = \textit{Cost of holding per unit of inventory} (Biaya persediaan per unit) -> \${0.30}
\end{itemize}

Maka, EOQ-nya adalah sebagai berikut.

\begin{equation}
    \begin{split}
		EOQ
		&= \sqrt{\frac{2 \times D \times Co}{Ch}} \\
		&= \sqrt{\frac{2 \times 2,400 \times 10}{0.30}} \\
		&= \sqrt{\frac{48,000}{0.30}} \\
		&= \sqrt{160,000} \\
		&= 400
    \end{split}
\end{equation}

Dapat dilihat bahwa EOQ dari material DX adalah sebesar 400 unit. Sekarang dapat dihitung berapa jumlah penjualan tahunan, biaya pemesanan tahunan, biaya penyimpaan tahunan, dan juga kombinasi dari biaya pemesanan tahunan dan biaya persediaan tahunan sebagai berikut.

\textbf{Jumlah penjualan tahunan}

= Kebutuhan tahunan / EOQ

= 2,400 unit / 400 unit

= 6 pesanan per tahun

\textbf{Biaya pemesanan tahunan}

= Jumlah penjualan tahunan * Biaya pemesanan per unit

= 6 pesanan * \${10}

= \${60}

\textbf{Biaya penyimpanan tahunan}

= Rata-rata unit * Biaya penyimpanan

= (400/2) * 0.3

= \${60}

\textbf{Kombinasi antara biaya pemesanan dan biaya penyimpaan}

= Biaya pemesanan tahunan + biaya penyimpanan tahunan

= \${60} + \${60}

= \${120}

MASUKKIN TABELNYA

Pada tabel diatas, dapat dilihat bahwa dengan data yang sama menghasilkan hitungan yang berbeda tergantung dari berapa banyak jumlah penjualan tahunannya.

Dari hitungan EOQ yang sudah dilakukan sebelumnya, jumlah penjualan tahunan sebesar 6 pesanan per tahun mendapatkan biaya kombinasi yang lebih sedikit dan stabil dibandingkan dengan kurang atau lebih dari 6 pesanan per tahunnya. Hal ini dikarenakan jika semakin kecil angka penjualan tahunannya maka hal tersebut akan berdampak pada tingginya biaya penyimpanan yang menyebabkan ketidakseimbangan antara biaya pemesanan dan biaya penyimpanan. Sementara itu, jika penjualan pertahunnya itu tinggi maka hal tersebut akan berdampak pada tingginya biaya pemesanan yang menyebabkan hal yang serupa.

Jadi, dapat disimpulkan bahwa perhitungan EOQ ini bersifat konstan terhadap biaya pemesanan dan biaya penyimpanan.

\section{ROP}

Dalam EOQ, ditentukan titik pemesanan kembali atau \textit{reorder point} yang biasa dikenal sebagai (ROP), yaitu jumlah persediaan tetap setiap kali pemesanan. ROP dilakukan bila persediaan bisa memenuhi kebutuhan produksi selama masa tenggang waktu. ROP menghendaki pengecekan kartu catatan secara teratur.

Untuk menentukan waktu pemesanan kembali atau \textit{reorder point} dapat dilakukan dengan rumus berikut.

ROP = (LT $\times$ AU) + SS

ROP = \textit{Reorder point}, yaitu tingkat dimana perusahaan harus memesan kembali.

LT = \textit{Leadtime}, yaitu masa kadaluarsa antara pemesanan sampai dengan kedatangan bahan.

AU = \textit{Average usage}, yaitu pemakaian rata-rata dalam pemakaian tertentu.

SS = \textit{Safety stok} yaitu besarnya persediaan atau bisa dibilang \textit{minimum inventory point}.

\section{Safety Stok}
\section{Analisa Persediaan}
\section{Pengendalian Sistem Persediaan}

