 %!TEX root = ./template-skripsi.tex
%-------------------------------------------------------------------------------
%                            BAB II
%               KAJIAN TEORI
%-------------------------------------------------------------------------------

\chapter{KAJIAN PUSTAKA} 

\section{Pengertian Inventarisasi}

Inventarisasi merupakan proses kegiatan pengelolaan persediaan barang yang dimiliki oleh instansi dalam melakukan kegiatan-kegiatannya. Jika dalam instansi tersebut tidak ada kegiatan pengelolaan barang, kemungkinan besar sistem yang ada di instansi tersebut tidak akan berjalan dengan baik.

Berdasarkan Kamus Besar Bahasa Indonesia (KBBI), Inventarisasi merupakan pencatatan atau pendaftaran barang-barang milik kantor, (sekolah, rumah tangga, dan sebagainya) yang dipakai dalam melaksanakan tugas.

\section{Jenis-jenis Inventarisasi}

Dalam Inventarisasi, terdapat banyak jenis-jenis yang tergantung pada sesuatu yang dijual. Beberapa diantaranya adalah:

\begin{enumerate}
    \item Barang jadi / barang untuk dijual (Finished goods/for-sale goods): Produk yang sudah siap untuk Anda jual ke pelanggan.
    \item Bahan baku (Raw Material): Persediaan yang Anda gunakan untuk membuat barang jadi.
    \item Pekerjaan dalam proses (Work-In-Progress): Pada dasarnya, barang yang belum jadi merupakan inventaris yang merupakan bagian dari proses manufaktur.
    \item Barang MRO (MRO Good): MRO adalah singkatan dari Maintaining (pemeliharaan), Repair (perbaikan) dan Operating (pengoperasian). Ini adalah inventaris yang Anda gunakan untuk mendukung proses pembuatan (proses produksi).
    \item Stok pengaman (Safety Stock): Persediaan tambahan yang Anda simpan untuk mengatasi kekurangan pemasok atau lonjakan permintaan.
\end{enumerate}

\section{Cara Kerja Inventarisasi}

Dalam proses mengatur inventaris, terdapat beberapa tahapan yang dilakukan seperti:

\begin{enumerate}
    \item Pembelian (Purchasing)
    
    Ini bisa berarti membeli bahan mentah untuk diubah menjadi produk, atau membeli produk untuk dijual tanpa perlu perakitan.

    \item Produksi (Production)
    
    Membuat produk jadi Anda dari bagian-bagian penyusunnya. Tidak setiap perusahaan akan terlibat dalam manufaktur – grosir, misalnya, mungkin melewatkan langkah ini sepenuhnya.

    \item Menyimpan bahan mentah dan jadi (Holding Stock)
    
    Menyimpan bahan mentah Anda sebelum diproduksi (jika diperlukan), dan barang jadi Anda sebelum dijual.

    \item Penjualan (Sales)
    
    Menjual barang Anda ke tangan pelanggan, dan menerima pembayaran.

    \item Pelaporan (Reporting)
    
    Bisnis perlu mengetahui berapa banyak yang dijual, dan berapa banyak uang yang dihasilkan dari setiap penjualan.

\end{enumerate}

\section{Prosedur Inventarisasi}


\section{Manfaat Inventarisasi}

Dengan melakukan manajemen inventaris, terdapat beberapa manfaat seperti:

\begin{itemize}
    \item Memudahkan untuk menambahkan produk dan saluran baru, menganalisis kinerja, dan memberdayakan tenaga penjualan dengan informasi produk terbaru, sehingga Anda dapat meningkatkan pendapatan penjualan.
    \item Menghilangkan inefisiensi yang menyebabkan kehilangan stok, kelebihan stok, dan kehabisan stok – mengurangi biaya penyimpanan dan meningkatkan margin.
    \item Kurangi waktu dan tenaga yang dihabiskan untuk admin inventaris, menghemat biaya staf.
\end{itemize}

\section{Tujuan Inventarisasi}

Proses Inventarisasi dilakukan dengan tujuan untuk membantu kelancaran administrasi perusahaan maupun instansi agar aset dapat terawasi dengan baik.

Beberapa tujuan dari pelaksanaan Inventarisasi adalah:

\begin{enumerate}
    \item Menjaga sarana prasarana yang dimiliki oleh sebuah instansi atau perusahaan,
    \item Memudahkan kegiatan kontrol inventaris terhadap penggunaan budget perusahaan.
    \item Menjadi bahan pertimbangan untuk pengadaan atau pemeliharaan.
    \item Membantu merencanakan, menyalurkan, memelihara dan menyimpan aset yang dimiliki instansi/perusahaan.
    \item Sebagai pedoman untuk menghitung nilai kekayaan aset.
    \item Mempercepat proses pembuatan laporan.
    \item Sebagai bahan rujukan jika terjadi kecurangan pegawai atau pencurian dalam perusahaan/instansi.
    \item Untuk memeriksa dan mengontrol keluar masuk barang, termasuk barang hibah/pemberian.
\end{enumerate}

https://www.jurnal.id/id/blog/pengertian-dan-pengelolaan-inventaris-adalah/
https://belajarekonomi.com/manajemen-inventaris/

% Kutipan https://rangkulteman.id/berita/inventaris-adalah-pengertian-dan-pengelolaannya-pada-sebuah-bisnis#1_Menurut_KBBI
% Kutipan https://www.jurnal.id/id/blog/pengertian-dan-pengelolaan-inventaris-adalah/
% Kutipan https://elib.unikom.ac.id/files/disk1/609/jbptunikompp-gdl-akrianhayy-30439-8-unikom_-i.pdf
% Kutipan https://kbbi.kemdikbud.go.id/entri/inventarisasi