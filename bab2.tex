 %!TEX root = ./template-skripsi.tex
%-------------------------------------------------------------------------------
%                            BAB II
%               KAJIAN TEORI
%-------------------------------------------------------------------------------

\chapter{KAJIAN PUSTAKA} 

\section{Pengertian Persediaan dan Manajemen Persediaan}

Persediaan adalah sebuah stok barang yang dimiliki oleh sebuah perusahaan. Persediaan dapat berupa bahan mentah, bahan baku, barang jadi, barang dalam proses, hingga bahan pembantu. Persediaan atau stok barang merupakan aset perusahaan yang berharga, karena hal ini berkaitan erat dengan proses produksi. Persediaan yang tidak terstruktur akan membuat perusahaan merugi, sehingga penting untuk menerapkan manajemen persediaan dalam sebuah bisnis atau usaha.

Manajemen persediaan adalah sebuah cara untuk melakukan pengawasan, kontrol, pengelolaan terhadap persediaan atau stok barang yang dimiliki oleh sebuah perusahaan. Segala bentuk kegiatan atau aktivitas yang berkaitan dengan memperoleh, menyimpan, hingga menggunakan persediaan merupakan bagian dari manajemen persediaan.

Manajemen persediaan memiliki beberapa fungsi, yaitu:
\begin{enumerate}
	\item Mencegah terjadinya kekurangan persediaan.
	\item Mencegah barang dari supplier tidak sesuai kebutuhan.
	\item Memastikan proses produksi berjalan dengan lancar.
	\item Mengantisipasi permintaan yang mendadak.
	\item Menyesuaikan pembelian dengan jadwal produksi.
\end{enumerate}

Selain beberapa fungsi yang sudah disebutkan diatas, Manajemen persediaan juga memiliki tujuan. Setiap manajemen yang dilakukan pasti memiliki tujuan yang ingin dicapai, beberapa tujuan dari Manajemen persediaan adalah sebagai berikut.
\begin{enumerate}
	\item Mengantisipasi kenaikan harga dari bahan baku.
	\item Memastikan stok atau persediaan barang selalu tersedia.
	\item Mengurangi resiko bahan baku yang datang terlambat.
	\item Menjaga jumlah persediaan yang ada di pasaran tetap stabil.
	\item Mengantisipasi kemungkinan adanya perubahan, baik dari segi penawaran maupun permintaan.
\end{enumerate}

\section{Jenis-jenis Manajemen Persediaan}

Manajemen persediaan dibagi menjadi beberapa jenis, diantaranya:

\begin{enumerate}
	\item Bahan Mentah
	
	Bahan mentah atau biasa yang disebut dengan bahan baku, merupakan bahan utama atau dasar dari dibuatnya suatu produk. Tanpa adanya bahan baku, maka produk yang dijual tidak akan bisa untuk diproduksi.

	Bahan mentah memiliki peran yang paling penting dalam memproduksi suatu barang/produk. Untuk itu, manajemen persediaan diperlukan dalam mengelola bahan baku agar bahan baku yang diperlukan selalu tersedia dan siap untuk diproses.
	
	\item Barang Setengah Jadi
	
	Barang setengah jadi atau bisa disebut sebagai barang dalam proses merupakan barang yang belum sepenuhnya bisa digunakan, sehingga perlu untuk diproses lebih lanjut untuk menjadi barang jadi, yang nantinya siap untuk digunakan.

	Manajemen persediaan berguna untuk menghitung besar serta banyaknya barang setengah jadi tersebut untuk memenuhi kebutuhan pasar.

	\item Barang Jadi
	
	Barang jadi merupakan bahan mentah yang diproses menjadi barang setengah jadi, lalu diproses kembali sehingga menjadi barang jadi. Barang jadi bisa dibilang barang yang sudah siap untuk dijual kepada konsumen.

	Manajemen persediaan berguna untuk mengatur pengiriman produk-produk tersebut ke pasar sehingga keadaan produk di pasar tetap stabil.
\end{enumerate}

\section{Biaya Persediaan}

Penetapan biaya persediaan atau evaluasi persediaan memungkinkan perusahaan untuk memberikan nilai moneter untuk barang-barang dalam persediaan mereka. Inventaris perusahaan seringkali merupakan aset terbesarnya dan pengukuran yang tepat untuk memastikan keakuratan laporan keuangan.

Untuk menentukan biaya persediaan, diperlukan lima langkah-langkah sebagai berikut.

\begin{enumerate}
	\item Menentukan periode waktu tertentu dimana Anda perlu menemukan nilai inventaris anda.
	\item Memastikan stok atau persediaan barang selalu tersedia.
	\item Mengurangi resiko bahan baku yang datang terlambat.
	\item Menjaga jumlah persediaan yang ada di pasaran tetap stabil.
	\item Mengantisipasi kemungkinan adanya perubahan, baik dari segi penawaran maupun permintaan.
\end{enumerate}

Dalam bisnis modern, terdapat tiga metode yang digunakan dalam menghitung biaya persediaan, yaitu:

\begin{enumerate}
	\item First In, First Out (FIFO)
	
	\textit{First-in, first-out} atau FIFO adalah metode dimana aset yang diproduksi dan diperoleh terlebih dahulu juga dijual atau digunakan terlebih dahulu. Saat menggunakan FIFO sebagai metode pilihan, gunakan perhitungan ini:
	
	
	\textbf{HPP = biaya persediaan terlama X jumlah persediaan yang terjual}

	Contoh FIFO:
	\begin{enumerate}
		\item Temukan unit yang tersedia untuk dijual
		\item Temukan jumlah unit yang terjual
		\item Temukan inventaris akhir
		\item Gunakan rumus FIFO
	\end{enumerate}

	\item Last In, First Out (LIFO)
	
	\textit{Last-in, first-out} atau LIFO adalah metode yang mencatat barang-barang yang baru saja diproduksi sebagai barang yang terjual lebih dulu. Saat menggunakan LIFO sebagai metode pilihan, gunakan perhitungan ini:


	\textbf{HPP = biaya persediaan terakhir X jumlah persediaan yang terjual}

	Contoh LIFO:
	\begin{enumerate}
		\item Tentukan biaya persediaan terbaru
		\item Temukan jumlah unit yang terjual
		\item Gunakan rumus LIFO
	\end{enumerate}

	\item Rata-rata tertimbang
	
	Rata-rata tertimbang atau biaya rata-rata tertimbang yang biasa dikenal sebagai \textit{Weighted Average Cost} (WAC) adalah metode yang menentukan jumlah masuk ke HPP dan persediaan melalui penggunaan rata-rata tertimbang. Saat menggunakan WAC, gunakan perhitungan ini:


	\textbf{WAC per unit = harga pokok barang yang tersedia / unit yang tersedia}

	contoh WAC:
	\begin{enumerate}
		\item Tentukan biaya setiap penjualan
		\item Tambahkan penjualan Anda bersama-sama
		\item Temukan unit yang tersedia untuk dijual
		\item Gunakan rumus rata-rata tertimbang
	\end{enumerate}

\end{enumerate}

\section{Perputaran Persediaan}
\section{EOQ}

\textit{Economic order quantity} (EOQ) merupakan 

\section{ROP}
\section{Safety Stok}
\section{Analisa Persediaan}
\section{Pengendalian Sistem Persediaan}

