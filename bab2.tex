 %!TEX root = ./template-skripsi.tex
%-------------------------------------------------------------------------------
%                            BAB II
%               KAJIAN TEORI
%-------------------------------------------------------------------------------

\chapter{KAJIAN PUSTAKA} 

\section{Pengertian Persediaan dan Manajemen Persediaan}

Persediaan adalah sebuah stok barang yang dimiliki oleh sebuah perusahaan. Persediaan dapat berupa bahan mentah, bahan baku, barang jadi, barang dalam proses, hingga bahan pembantu. Persediaan atau stok barang merupakan aset perusahaan yang berharga, karena hal ini berkaitan erat dengan proses produksi. Persediaan yang tidak terstruktur akan membuat perusahaan merugi, sehingga penting untuk menerapkan manajemen persediaan dalam sebuah bisnis atau usaha.

Manajemen persediaan adalah sebuah cara untuk melakukan pengawasan, kontrol, pengelolaan terhadap persediaan atau stok barang yang dimiliki oleh sebuah perusahaan. Segala bentuk kegiatan atau aktivitas yang berkaitan dengan memperoleh, menyimpan, hingga menggunakan persediaan merupakan bagian dari manajemen persediaan.

Manajemen persediaan memiliki beberapa fungsi, yaitu:
\begin{enumerate}
	\item Mencegah terjadinya kekurangan persediaan.
	\item Mencegah barang dari supplier tidak sesuai kebutuhan.
	\item Memastikan proses produksi berjalan dengan lancar.
	\item Mengantisipasi permintaan yang mendadak.
	\item Menyesuaikan pembelian dengan jadwal produksi.
\end{enumerate}

Selain beberapa fungsi yang sudah disebutkan diatas, Manajemen persediaan juga memiliki tujuan. Setiap manajemen yang dilakukan pasti memiliki tujuan yang ingin dicapai, beberapa tujuan dari Manajemen persediaan adalah sebagai berikut.
\begin{enumerate}
	\item Mengantisipasi kenaikan harga dari bahan baku.
	\item Memastikan stok atau persediaan barang selalu tersedia.
	\item Mengurangi resiko bahan baku yang datang terlambat.
	\item Menjaga jumlah persediaan yang ada di pasaran tetap stabil.
	\item Mengantisipasi kemungkinan adanya perubahan, baik dari segi penawaran maupun permintaan.
\end{enumerate}

\section{Jenis-jenis Manajemen Persediaan}

Manajemen persediaan dibagi menjadi beberapa jenis, diantaranya:

\begin{enumerate}
	\item Bahan Mentah
	
	Bahan mentah atau biasa yang disebut dengan bahan baku, merupakan bahan utama atau dasar dari dibuatnya suatu produk. Tanpa adanya bahan baku, maka produk yang dijual tidak akan bisa untuk diproduksi.

	Bahan mentah memiliki peran yang paling penting dalam memproduksi suatu barang/produk. Untuk itu, manajemen persediaan diperlukan dalam mengelola bahan baku agar bahan baku yang diperlukan selalu tersedia dan siap untuk diproses.
	
	\item Barang Setengah Jadi
	
	Barang setengah jadi atau bisa disebut sebagai barang dalam proses merupakan barang yang belum sepenuhnya bisa digunakan, sehingga perlu untuk diproses lebih lanjut untuk menjadi barang jadi, yang nantinya siap untuk digunakan.

	Manajemen persediaan berguna untuk menghitung besar serta banyaknya barang setengah jadi tersebut untuk memenuhi kebutuhan pasar.

	\item Barang Jadi
	
	Barang jadi merupakan bahan mentah yang diproses menjadi barang setengah jadi, lalu diproses kembali sehingga menjadi barang jadi. Barang jadi bisa dibilang barang yang sudah siap untuk dijual kepada konsumen.

	Manajemen persediaan berguna untuk mengatur pengiriman produk-produk tersebut ke pasar sehingga keadaan produk di pasar tetap stabil.
\end{enumerate}

\section{Biaya Persediaan}

Penetapan biaya persediaan atau evaluasi persediaan memungkinkan perusahaan untuk memberikan nilai moneter untuk barang-barang dalam persediaan mereka. Inventaris perusahaan seringkali merupakan aset terbesarnya dan pengukuran yang tepat untuk memastikan keakuratan laporan keuangan.

Untuk menentukan biaya persediaan, diperlukan lima langkah-langkah sebagai berikut.

\begin{enumerate}
	\item Menentukan periode waktu tertentu dimana Anda perlu menemukan nilai inventaris anda.
	\item Memastikan stok atau persediaan barang selalu tersedia.
	\item Mengurangi resiko bahan baku yang datang terlambat.
	\item Menjaga jumlah persediaan yang ada di pasaran tetap stabil.
	\item Mengantisipasi kemungkinan adanya perubahan, baik dari segi penawaran maupun permintaan.
\end{enumerate}

Dalam bisnis modern, terdapat tiga metode yang digunakan dalam menghitung biaya persediaan, yaitu:

\begin{enumerate}
	\item First In, First Out (FIFO)
	
	\textit{First-in, first-out} atau FIFO adalah metode dimana aset yang diproduksi dan diperoleh terlebih dahulu juga dijual atau digunakan terlebih dahulu. Saat menggunakan FIFO sebagai metode pilihan, gunakan perhitungan ini:
	
	
	\textbf{HPP = biaya persediaan terlama X jumlah persediaan yang terjual}

	Contoh FIFO:
	\begin{enumerate}
		\item Temukan unit yang tersedia untuk dijual
		\item Temukan jumlah unit yang terjual
		\item Temukan inventaris akhir
		\item Gunakan rumus FIFO
	\end{enumerate}

	\item Last In, First Out (LIFO)
	
	\textit{Last-in, first-out} atau LIFO adalah metode yang mencatat barang-barang yang baru saja diproduksi sebagai barang yang terjual lebih dulu. Saat menggunakan LIFO sebagai metode pilihan, gunakan perhitungan ini:


	\textbf{HPP = biaya persediaan terakhir X jumlah persediaan yang terjual}

	Contoh LIFO:
	\begin{enumerate}
		\item Tentukan biaya persediaan terbaru
		\item Temukan jumlah unit yang terjual
		\item Gunakan rumus LIFO
	\end{enumerate}

	\item Rata-rata tertimbang
	
	Rata-rata tertimbang atau biaya rata-rata tertimbang yang biasa dikenal sebagai \textit{Weighted Average Cost} (WAC) adalah metode yang menentukan jumlah masuk ke HPP dan persediaan melalui penggunaan rata-rata tertimbang. Saat menggunakan WAC, gunakan perhitungan ini:


	\textbf{WAC per unit = harga pokok barang yang tersedia / unit yang tersedia}

	contoh WAC:
	\begin{enumerate}
		\item Tentukan biaya setiap penjualan
		\item Tambahkan penjualan Anda bersama-sama
		\item Temukan unit yang tersedia untuk dijual
		\item Gunakan rumus rata-rata tertimbang
	\end{enumerate}

\end{enumerate}

\section{Perputaran Persediaan}
\section{EOQ}

\textit{Economic order quantity} (EOQ) merupakan jumlah persediaan yang digunakan untuk meminimalkan jumlah dan biaya pemesanan yang terkait dengan bahan baku atau persediaan barang dagangan. Intinya, EOQ merupakan \textit{set point} yang dibuat dan digunakan untuk menjadi acuan dalam membantu perusahaan meminimalkan total biaya persediaan.

Dua faktor penting yang menjadi penentu dalam menentukan \textit{economic order quantity} (EOQ) adalah biaya pemesanan dan biaya penyimpanan.

\begin{enumerate}
	\item Biaya pemesanan
	
	Biaya pemesanan merupakan biaya yang dikeluarkan setiap pesanan. Contoh hal yang termasuk biaya pemesanan adalah biaya pengiriman, biaya pemrosesan pembayaran, dan lain-lain.

	\item Biaya persediaan
	
	Biaya persediaan merupakan biaya yang dikeluarkan untuk menyimpan persediaan di toko atau gudang. Contoh hal yang termasuk dalam biaya penyimpaan adalah biaya sewa ruang penyimpanan, pajak properti, dan lain-lain.
\end{enumerate}

Formula atau rumus yang digunakan untuk menentukan EOQ adalah:

\[EOQ=\sqrt{\frac{2 \times D \times Co}{Ch}}\]

\begin{itemize}
	\item D = \textit{Demand per year} (Kebutuhan per tahun)
	\item Co = \textit{Cost per order} (Biaya per pesanan)
	\item Ch = \textit{Cost of holding per unit of inventory} (Biaya persediaan per unit)
\end{itemize}

\textbf{Contoh Kasus}

Sebuah material DX digunakan rutin setiap tahunnya. Data kebutuhan per tahun, biaya pemesanan, dan biaya persediaan per unit adalah sebagai berikut.

\begin{itemize}
	\item Kebutuhan tahunan = 2,400 unit
	\item Biaya per pesanan = \${10} per pesanan
	\item Biaya persediaan per unit = \${0.30} per unit
\end{itemize}

Diketahui:
\begin{itemize}
	\item D = \textit{Demand per year} (Kebutuhan per tahun) -> 2,400
	\item Co = \textit{Cost per order} (Biaya per pesanan) -> \${10}
	\item Ch = \textit{Cost of holding per unit of inventory} (Biaya persediaan per unit) -> \${0.30}
\end{itemize}

Maka, EOQ-nya adalah sebagai berikut.

\begin{equation}
    \begin{split}
		EOQ
		&= \sqrt{\frac{2 \times D \times Co}{Ch}} \\
		&= \sqrt{\frac{2 \times 2,400 \times 10}{0.30}} \\
		&= \sqrt{\frac{48,000}{0.30}} \\
		&= \sqrt{160,000} \\
		&= 400
    \end{split}
\end{equation}

Dapat dilihat bahwa EOQ dari material DX adalah sebesar 400 unit. Sekarang dapat dihitung berapa jumlah penjualan tahunan, biaya pemesanan tahunan, biaya penyimpaan tahunan, dan juga kombinasi dari biaya pemesanan tahunan dan biaya persediaan tahunan sebagai berikut.

\textbf{Jumlah penjualan tahunan}

= Kebutuhan tahunan / EOQ

= 2,400 unit / 400 unit

= 6 pesanan per tahun

\textbf{Biaya pemesanan tahunan}

= Jumlah penjualan tahunan * Biaya pemesanan per unit

= 6 pesanan * \${10}

= \${60}

\textbf{Biaya penyimpanan tahunan}

= Rata-rata unit * Biaya penyimpanan

= (400/2) * 0.3

= \${60}

\textbf{Kombinasi antara biaya pemesanan dan biaya penyimpaan}

= Biaya pemesanan tahunan + biaya penyimpanan tahunan

= \${60} + \${60}

= \${120}

MASUKKIN TABELNYA

Pada tabel diatas, dapat dilihat bahwa dengan data yang sama menghasilkan hitungan yang berbeda tergantung dari berapa banyak jumlah penjualan tahunannya.

Dari hitungan EOQ yang sudah dilakukan sebelumnya, jumlah penjualan tahunan sebesar 6 pesanan per tahun mendapatkan biaya kombinasi yang lebih sedikit dan stabil dibandingkan dengan kurang atau lebih dari 6 pesanan per tahunnya. Hal ini dikarenakan jika semakin kecil angka penjualan tahunannya maka hal tersebut akan berdampak pada tingginya biaya penyimpanan yang menyebabkan ketidakseimbangan antara biaya pemesanan dan biaya penyimpanan. Sementara itu, jika penjualan pertahunnya itu tinggi maka hal tersebut akan berdampak pada tingginya biaya pemesanan yang menyebabkan hal yang serupa.

Jadi, dapat disimpulkan bahwa perhitungan EOQ ini bersifat konstan terhadap biaya pemesanan dan biaya penyimpanan.

\section{ROP}
\section{Safety Stok}
\section{Analisa Persediaan}
\section{Pengendalian Sistem Persediaan}

