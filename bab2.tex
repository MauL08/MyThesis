 %!TEX root = ./template-skripsi.tex
%-------------------------------------------------------------------------------
%                            BAB II
%               KAJIAN TEORI
%-------------------------------------------------------------------------------

\chapter{KAJIAN PUSTAKA} 

\section{Pengertian Persediaan dan Manajemen Persediaan}

Persediaan adalah sebuah stok barang yang dimiliki oleh sebuah perusahaan. Persediaan dapat berupa bahan mentah, bahan baku, barang jadi, barang dalam proses, hingga bahan pembantu. Persediaan atau stok barang merupakan aset perusahaan yang berharga, karena hal ini berkaitan erat dengan proses produksi. Persediaan yang tidak terstruktur akan membuat perusahaan merugi, sehingga penting untuk menerapkan manajemen persediaan dalam sebuah bisnis atau usaha.

Manajemen persediaan adalah sebuah cara untuk melakukan pengawasan, kontrol, pengelolaan terhadap persediaan atau stok barang yang dimiliki oleh sebuah perusahaan. Segala bentuk kegiatan atau aktivitas yang berkaitan dengan memperoleh, menyimpan, hingga menggunakan persediaan merupakan bagian dari manajemen persediaan.

Manajemen persediaan memiliki beberapa fungsi, yaitu:
\begin{enumerate}
	\item Mencegah terjadinya kekurangan persediaan.
	\item Mencegah barang dari supplier tidak sesuai kebutuhan.
	\item Memastikan proses produksi berjalan dengan lancar.
	\item Mengantisipasi permintaan yang mendadak.
	\item Menyesuaikan pembelian dengan jadwal produksi.
\end{enumerate}

Selain beberapa fungsi yang sudah disebutkan diatas, Manajemen persediaan juga memiliki tujuan. Setiap manajemen yang dilakukan pasti memiliki tujuan yang ingin dicapai, beberapa tujuan dari Manajemen persediaan adalah sebagai berikut.
\begin{enumerate}
	\item Mengantisipasi kenaikan harga dari bahan baku.
	\item Memastikan stok atau persediaan barang selalu tersedia.
	\item Mengurangi resiko bahan baku yang datang terlambat.
	\item Menjaga jumlah persediaan yang ada di pasaran tetap stabil.
	\item Mengantisipasi kemungkinan adanya perubahan, baik dari segi penawaran maupun permintaan.
\end{enumerate}

\section{Jenis-jenis Manajemen Persediaan}

Manajemen persediaan dibagi menjadi beberapa jenis, diantaranya:

\begin{enumerate}
	\item Bahan Mentah
	
	Bahan mentah atau biasa yang disebut dengan bahan baku, merupakan bahan utama atau dasar dari dibuatnya suatu produk. Tanpa adanya bahan baku, maka produk yang dijual tidak akan bisa untuk diproduksi.

	Bahan mentah memiliki peran yang paling penting dalam memproduksi suatu barang/produk. Untuk itu, manajemen persediaan diperlukan dalam mengelola bahan baku agar bahan baku yang diperlukan selalu tersedia dan siap untuk diproses.
	
	\item Barang Setengah Jadi
	
	Barang setengah jadi atau bisa disebut sebagai barang dalam proses merupakan barang yang belum sepenuhnya bisa digunakan, sehingga perlu untuk diproses lebih lanjut untuk menjadi barang jadi, yang nantinya siap untuk digunakan.

	Manajemen persediaan berguna untuk menghitung besar serta banyaknya barang setengah jadi tersebut untuk memenuhi kebutuhan pasar.

	\item Barang Jadi
	
	Barang jadi merupakan bahan mentah yang diproses menjadi barang setengah jadi, lalu diproses kembali sehingga menjadi barang jadi. Barang jadi bisa dibilang barang yang sudah siap untuk dijual kepada konsumen.

	Manajemen persediaan berguna untuk mengatur pengiriman produk-produk tersebut ke pasar sehingga keadaan produk di pasar tetap stabil.
\end{enumerate}

\section{Jenis Biaya dalam Persediaan}



\section{Perputaran Persediaan}
\section{EOQ}
\section{ROP}
\section{Safety Stok}
\section{Analisa Persediaan}
\section{Pengendalian Sistem Persediaan}

