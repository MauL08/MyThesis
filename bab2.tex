 %!TEX root = ./template-skripsi.tex
%-------------------------------------------------------------------------------
%                            BAB II
%               KAJIAN TEORI
%-------------------------------------------------------------------------------

\chapter{KAJIAN PUSTAKA} 

\section{Inventarisasi}

Inventarisasi merupakan proses kegiatan pengelolaan persediaan barang yang dimiliki oleh instansi dalam melakukan kegiatan-kegiatannya. Jika dalam instansi tersebut tidak ada kegiatan pengelolaan barang, kemungkinan besar sistem yang ada di instansi tersebut tidak akan berjalan dengan baik.

Berdasarkan Kamus Besar Bahasa Indonesia (KBBI), Inventarisasi merupakan pencatatan atau pendaftaran barang-barang milik kantor, (sekolah, rumah tangga, dan sebagainya) yang dipakai dalam melaksanakan tugas.

Beberapa pendapat para ahli yang berhubungan dengan inventarisasi adalah sebagai berikut:
\begin{enumerate}
	\item Ahli 1
	\item Ahli 2
	\item Ahli 3
\end{enumerate}
% Menurut Soemarsono SR., inventaris artinya merupakan keseluruhan barang yang dimanfaatkan kantor dan disertai dengan kondisi barang, jenisnya, harga juga jumlahnya.

Inventarisasi mempunyai manfaat sebagai 

% Kutipan https://rangkulteman.id/berita/inventaris-adalah-pengertian-dan-pengelolaannya-pada-sebuah-bisnis#1_Menurut_KBBI
% Kutipan https://www.jurnal.id/id/blog/pengertian-dan-pengelolaan-inventaris-adalah/
% Kutipan https://elib.unikom.ac.id/files/disk1/609/jbptunikompp-gdl-akrianhayy-30439-8-unikom_-i.pdf
% Kutipan https://kbbi.kemdikbud.go.id/entri/inventarisasi