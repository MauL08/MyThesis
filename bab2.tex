 %!TEX root = ./template-skripsi.tex
%-------------------------------------------------------------------------------
%                            BAB II
%               KAJIAN TEORI
%-------------------------------------------------------------------------------

\chapter{KAJIAN PUSTAKA} 

\section{Pengertian Persediaan dan Manajemen Persediaan}

Pada buku Dasar-Dasar Manajemen \citep{dasarmanajemen}, dijelaskan bahwa persediaan adalah sebuah stok barang yang dimiliki oleh sebuah perusahaan. Persediaan dapat berupa bahan mentah, bahan baku, barang jadi, barang dalam proses, hingga bahan pembantu. Persediaan atau stok barang merupakan aset perusahaan yang berharga, karena hal ini berkaitan erat dengan proses produksi. Persediaan yang tidak terstruktur akan membuat perusahaan merugi, sehingga penting untuk menerapkan manajemen persediaan dalam sebuah bisnis atau usaha.

Manajemen persediaan adalah sebuah cara untuk melakukan pengawasan, kontrol, pengelolaan terhadap persediaan atau stok barang yang dimiliki oleh sebuah perusahaan. Segala bentuk kegiatan atau aktivitas yang berkaitan dengan memperoleh, menyimpan, hingga menggunakan persediaan merupakan bagian dari manajemen persediaan.

Manajemen persediaan memiliki beberapa fungsi, yaitu:
\begin{enumerate}
	\item Mencegah terjadinya kekurangan persediaan.
	\item Mencegah barang dari supplier tidak sesuai kebutuhan.
	\item Memastikan proses produksi berjalan dengan lancar.
	\item Mengantisipasi permintaan yang mendadak.
	\item Menyesuaikan pembelian dengan jadwal produksi.
\end{enumerate}

Selain beberapa fungsi yang sudah disebutkan diatas, Manajemen persediaan juga memiliki tujuan. Setiap manajemen yang dilakukan pasti memiliki tujuan yang ingin dicapai, beberapa tujuan dari Manajemen persediaan adalah sebagai berikut.
\begin{enumerate}
	\item Mengantisipasi kenaikan harga dari bahan baku.
	\item Memastikan stok atau persediaan barang selalu tersedia.
	\item Mengurangi resiko bahan baku yang datang terlambat.
	\item Menjaga jumlah persediaan yang ada di pasaran tetap stabil.
	\item Mengantisipasi kemungkinan adanya perubahan, baik dari segi penawaran maupun permintaan.
\end{enumerate}

\section{Jenis-jenis Manajemen Persediaan}

Manajemen persediaan dibagi menjadi beberapa jenis, diantaranya:

\begin{enumerate}
	\item Bahan Mentah
	
	Bahan mentah atau biasa yang disebut dengan bahan baku, merupakan bahan utama atau dasar dari dibuatnya suatu produk. Tanpa adanya bahan baku, maka produk yang dijual tidak akan bisa untuk diproduksi.

	Bahan mentah memiliki peran yang paling penting dalam memproduksi suatu barang/produk. Untuk itu, manajemen persediaan diperlukan dalam mengelola bahan baku agar bahan baku yang diperlukan selalu tersedia dan siap untuk diproses.
	
	\item Barang Setengah Jadi
	
	Barang setengah jadi atau bisa disebut sebagai barang dalam proses merupakan barang yang belum sepenuhnya bisa digunakan, sehingga perlu untuk diproses lebih lanjut untuk menjadi barang jadi, yang nantinya siap untuk digunakan.

	Manajemen persediaan berguna untuk menghitung besar serta banyaknya barang setengah jadi tersebut untuk memenuhi kebutuhan pasar.

	\item Barang Jadi
	
	Barang jadi merupakan bahan mentah yang diproses menjadi barang setengah jadi, lalu diproses kembali sehingga menjadi barang jadi. Barang jadi bisa dibilang barang yang sudah siap untuk dijual kepada konsumen.

	Manajemen persediaan berguna untuk mengatur pengiriman produk-produk tersebut ke pasar sehingga keadaan produk di pasar tetap stabil.
\end{enumerate}

\section{Biaya Persediaan}

Penetapan biaya persediaan atau evaluasi persediaan memungkinkan perusahaan untuk memberikan nilai moneter untuk barang-barang dalam persediaan mereka. Inventaris perusahaan seringkali merupakan aset terbesarnya dan pengukuran yang tepat untuk memastikan keakuratan laporan keuangan.

Untuk menentukan biaya persediaan, diperlukan lima langkah-langkah sebagai berikut.

\begin{enumerate}
	\item Menentukan periode waktu tertentu yang dimana perlu menemukan nilai inventaris.
	\item Memastikan stok atau persediaan barang selalu tersedia.
	\item Mengurangi resiko bahan baku yang datang terlambat.
	\item Menjaga jumlah persediaan yang ada di pasaran tetap stabil.
	\item Mengantisipasi kemungkinan adanya perubahan, baik dari segi penawaran maupun permintaan.
\end{enumerate}

Pada buku Pengendalian Persediaan \citep{pengendalianpersediaan}, dijelaskan bahwa terdapat beberapa metode yang digunakan untuk menentukan biaya persediaan, antara lain:

\subsection{Metode identifikasi khusus (\textit{Specific Identification Method})}

Dalam metode ini, setiap jenis barang yang ada di gudang harus diberi tanda sesuai dengan harga pokok per satuan barang tersebut dibeli. Jika terdapat barang yang harga satuannya berbeda dengan harga satuan barang yang ada di gudang, maka harus dipisahkan penyimpanannya dan diberi tanda pada harga berapa barang tersebut dibeli.

Masalah yang ada pada metode ini terletak dalam penyimpanan barang di gudang. Meskipun jenis barangnya sama, namun jika harga pokok per satuannya berbeda maka barang tersebut harus disimpan secara terpisah agar mudah diidentifikasi saat pemakaian nanti.

\subsection{Metode FIFO (\textit{First-in, First-out})}

Metode FIFO merupakan metode penentuan biaya persediaan dengan anggapan bahwa harga pokok per satuan barang yang pertama masuk dalam gudang digunakan untuk menentukan harga barang yang pertama kali dipakai. Sebagai contoh dapat dilihat perhitungan berikut.

Data mengenai barang saat minggu pertama bulan Januari 2020 sebagai berikut.

	\begin{table}[H]	
		\begin{center}
			\caption{Data tabel pada barang bulan Januari 2020}
			\label{tab:table1}
			\begin{tabular}{c|c} % <-- Alignments: 1st column left, 2nd middle and 3rd right, with vertical lines in between
			\textbf{Tanggal} & \textbf{Deskripsi} \\
			\hline
			01 Januari 2020 & Persediaan 8,000 kg dengan harga Rp1.000,00/kg \\
			08 Januari 2020 & Melakukan pembelian barang sebesar 12,000 kg \\
			&  dengan harga Rp1.200,00/kg \\
			09 Januari 2020 & Masuk proses produksi sebanyak 15,000 kg \\
			\end{tabular}
		\end{center}
	\end{table}

Barang yang masuk pertama yaitu barang yang pertama kali digunakan dalam proses produksi. Berdasarkan data pada tabel diatas, dapat dihitung biaya persediaannya dengan cara dibawah ini. 

8,000 kg $\times$ Rp1.000,00 = Rp8.000.000,00

7,000 kg $\times$ Rp1.200,00 = Rp8.400.000,00

Total = 15.000 kg = Rp16.400.000,00

\textbf{Biaya Persediaan akhir} = 5.000 kg $\times$ Rp1.200,00 = Rp6.000.000,00

\subsection{Metode LIFO (\textit{Last-in, Last-out})}

Metode LIFO merupakan metode penentuan biaya persediaan dengan anggapan bahwa harga pokok per satuan barang yang terakhir masuk dalam persediaan dipakai untuk menentukan harga pokok barang yang pertama kali dipakai dalam produksi. Sebagai contoh dapat dilihat perhitungan dengan data tabel yang sama.

Barang yang terakhir masuk merupakan barang yang digunakan terlebih dahulu dalam proses produksi.

12,000 kg $\times$ Rp1.200,00 = Rp14.400.000,00

3,000 kg $\times$ Rp1.000,00 = Rp3.000.000,00

Total = 15.000 kg = Rp17.400.000,00

\textbf{Biaya Persediaan akhir} = 5.000 kg $\times$ Rp1.000,00 = Rp5.000.000,00

\subsection{Metode rata-rata bergerak (\textit{Moving Average Method})}

Dalam metode ini, persediaan barang yang ada digudang dihitung harga pokok rata-ratanya dengan cara membagi total pokok dengan jumlah satuannya. Metode ini disebut juga rata-rata tertimbang, karena dalam menghitung rata-rata harga pokok persediaan barang, metode ini menggunakan kuantitas barang sebagai angka penimbangnya. Sebagai contoh dapat dilihat perhitungan dibawah dengan tabel yang sama seperti sebelumnya.

Biaya barang yang dipakai dalam proses produksi yaitu hasil kali kuantitas barang yang dipakai dan harga rata-rata per satuan.

8,000 kg $\times$ Rp1.000,00 = Rp8.000.000,00

12,000 kg $\times$ Rp1.200,00 = Rp14.000.000,00

Total = 20.000 kg = Rp22.400.000,00

Harga rata-rata = Rp22.400.000,00 : 20.000 kg = Rp1.120,00

Total = 15.000 kg $\times$ Rp1.120,00 = Rp16.800.000,00

\textbf{Biaya Persediaan akhir} = 5.000 kg $\times$ Rp1.120,00 = Rp5.600.000,00

\subsection{Metode biaya standar}

Pada metode ini, barang yang dibeli dicatat dalam kartu persediaan sebesar harga standar yaitu harga taksiran yang mencerminkan harga yang diharapkan akan terjadi dimasa yang akan datang. Harga standar merupakan harga yang diperkirakan untuk tahun tertentu.

\subsection{Metode rata-rata harga pokok bahan pada akhir bulan}

Dengan metode ini, pada akhir bulan dilakukan perhitungan harga pokok rata-rata per satuan tiap jenis persediaan barang yang ada digudang. Harga pokok rata-rata per satuan kemudian digunakan untuk menghitung harga pokok barang yang dipakai dalam produksi bulan berikutnya.

% \subsection{First In, First Out (FIFO)} 
	
% 	\textit{First-in, first-out} atau FIFO adalah metode dimana aset yang diproduksi dan diperoleh terlebih dahulu juga dijual atau digunakan terlebih dahulu. Saat menggunakan FIFO sebagai metode pilihan, gunakan perhitungan ini untuk menentukan harga pokok penjualan (HPP)-nya:
% 	\begin{equation}
% 		\begin{split}
% 			HPP
% 			&= Biaya\;persediaan\;terlama \times Jumlah\;persediaan\;yang\;terjual
% 		\end{split}
% 	\end{equation}

% 	\textbf{Contoh FIFO} : 

% 	\begin{table}[H]
% 		\begin{center}
% 		  \caption{Data tabel dalam persediaan barang}
% 		  \label{tab:table1}
% 		  \begin{tabular}{l|c|r} % <-- Alignments: 1st column left, 2nd middle and 3rd right, with vertical lines in between
% 			\textbf{Tanggal} & \textbf{Deskripsi} & \textbf{Unit dan Biaya} \\
% 			\hline
% 			1 Maret & Inventaris awal & 65 unit, \${15} per unit\\
% 			5 Maret & Beli &  130 unit, \${10} per unit\\
% 			10 Maret & Jual &  95 unit, \${20} per unit\\
% 			12 Maret & Beli &  35 unit, \${15} per unit\\
% 			15 Maret & Beli &  75 unit, \${20} per unit\\
% 			21 Maret & Jual &  115 unit, \${30} per unit\\
% 			28 Maret & Jual &  60 unit, \${35} per unit\\
% 		  \end{tabular}
% 		\end{center}
% 	  \end{table}
% 	  \begin{enumerate}
% 		\item \textbf{Tentukan jumlah unit yang tersedia untuk dijual}
		
% 		Tentukan jumlah unit yang tersedia untuk dijual dengan menjumlahkan semua unit beli dengan inventaris awal.

% 		\textbf{Unit yang tersedia untuk dijual = 65 + 130 + 35 + 75}

% 		\textbf{Unit yang tersedia untuk dijual = 305}

% 		Terdapat 305 unit yang tersedia untuk dijual.

% 		\item \textbf{Tentukan jumlah unit yang dijual}
		
% 		Tentukan jumlah unit yang dijual dengan menjumlahkan semua unit yang dijual.

% 		\textbf{Unit dijual = 95 + 115 + 60}

% 		\textbf{Unit dijual = 270}

% 		Ada 270 unit yang dijual.

% 		\item \textbf{Tentukan nilai inventaris akhir}
		
% 		Tentukan nilai inventaris akhir dengan cara mengurangi jumlah unit yang tersedia untuk dijual dengan jumlah unit yang dijual.

% 		\textbf{Inventaris akhir = Jumlah unit yang tersedia untuk dijual - jumlah unit yang dijual}

% 		\textbf{Inventaris akhir = 305 - 270}

% 		\textbf{Inventaris akhir = 35}

% 		Ada 35 unit yang tersisa pada inventaris akhir.

% 		\item \textbf{Menentukan HPP menggunakan rumus FIFO}
		
% 		Biaya persediaan terlama dapat ditentukan dengan nilai inventaris awal. Nilai tersebut didapat dengan mengalikan persediaan awal dan harga per unitnya. 

% 		\textbf{HPP = biaya persediaan terlama $\times$ jumlah persediaan yang dijual}

% 		\textbf{HPP = (65 $\times$ 15) $\times$ 270}

% 		\textbf{HPP = 263,250}

% 		Biaya barang yang dijual adalah \${263,250}.

% 	  \end{enumerate}

% \subsection{Last In, First Out (LIFO)}
	
% 	\textit{Last-in, first-out} atau LIFO adalah metode yang mencatat barang-barang yang baru saja diproduksi sebagai barang yang terjual lebih dulu. Saat menggunakan LIFO sebagai metode pilihan, gunakan perhitungan ini untuk menentukan harga pokok penjualan (HPP)-nya:
% 	\begin{equation}
% 		\begin{split}
% 			HPP
% 			&= Biaya\;persediaan\;terakhir \times Jumlah\;persediaan\;yang\;terjual
% 		\end{split}
% 	\end{equation}

% 	\textbf{Contoh LIFO} :

% 	\begin{table}[H]
% 		\begin{center}
% 		  \caption{Data tabel dalam persediaan barang}
% 		  \label{tab:table2}
% 		  \begin{tabular}{l|c|r} % <-- Alignments: 1st column left, 2nd middle and 3rd right, with vertical lines in between
% 			\textbf{Tanggal} & \textbf{Deskripsi} & \textbf{Unit dan Biaya} \\
% 			\hline
% 			3 April & Inventaris awal & 70 unit, \${15} per unit\\
% 			6 April & Beli &  125 unit, \${10} per unit\\
% 			10 April & Beli &  90 unit, \${20} per unit\\
% 			13 April & Jual &  40 unit, \${15} per unit\\
% 			16 April & Beli &  50 unit, \${15} per unit\\
% 			23 April & Jual &  100 unit, \${20} per unit\\
% 			29 April & Jual &  70 unit, \${20} per unit\\
% 		  \end{tabular}
% 		\end{center}
% 	  \end{table}

% 	\begin{enumerate}
% 		\item \textbf{Menentukan biaya persediaan terbaru}
		
% 		Untuk mencari biaya persediaan terbaru dapat dilakukan dengan menggunakan data pada inventaris awal.

% 		\textbf{Biaya persediaan terbaru = 70 $\times$ 15}

% 		\textbf{Biaya persediaan terbaru = 1,050}

% 		Biaya persediaan terbaru adalah \${1,050}.

% 		\item \textbf{Temukan jumlah unit yang dijual}
		
% 		Temukan jumlah unit yang terjual dengan menambahkan semua unit yang dijual.

% 		\textbf{Jumlah unit yang dijual = 40 + 100 + 70}

% 		\textbf{Jumlah unit yang dijual = 210}

% 		Ada 210 unit yang dijual.

% 		\item Gunakan rumus LIFO
		
% 		Gunakan biaya persediaan terbaru dan total unit yang dijual untuk menentukan nilai HPP-nya.

% 		\textbf{HPP = biaya persediaan terbaru $\times$ jumlah persediaan yang dijual}

% 		\textbf{HPP = 1,050 $\times$ 210}

% 		\textbf{HPP = 220,500}

% 		Biaya barang yang dijual adalah \${220,500}.

% 	\end{enumerate}

% 	\subsection{Rata-rata tertimbang}
	
% 	Rata-rata tertimbang atau biaya rata-rata tertimbang yang biasa dikenal sebagai \textit{Weighted Average Cost} (WAC) adalah metode yang menentukan jumlah masuk ke HPP dan persediaan melalui penggunaan rata-rata tertimbang. Saat menggunakan WAC, gunakan perhitungan ini:
% 	\begin{equation}
% 		\begin{split}
% 			WAC\;per\;unit
% 			&= Harga\;pokok\;barang\;yang\;tersedia \div unit\;yang\;tersedia
% 		\end{split}
% 	\end{equation}

% 	\textbf{Contoh WAC} :
% 	\begin{table}[H]
% 		\begin{center}
% 		  \caption{Data tabel dalam persediaan barang}
% 		  \label{tab:table3}
% 		  \begin{tabular}{l|c|r} % <-- Alignments: 1st column left, 2nd middle and 3rd right, with vertical lines in between
% 			\textbf{Tanggal} & \textbf{Deskripsi} & \textbf{Unit dan Biaya} \\
% 			\hline
% 			3 Mei & Inventaris awal & 30 unit, \${15} per unit\\
% 			6 Mei & Beli &  50 unit, \${15} per unit\\
% 			11 Mei & Jual &  25 unit, \${25} per unit\\
% 			15 Mei & Jual &  30 unit, \${10} per unit\\
% 			18 Mei & Beli &  15 unit, \${25} per unit\\
% 			22 Mei & Beli &  25 unit, \${35} per unit\\
% 			25 Mei & Jual &  35 unit, \${30} per unit\\
% 		  \end{tabular}
% 		\end{center}
% 	  \end{table}

% 	\begin{enumerate}
% 		\item Tentukan biaya setiap penjualan
		
% 		Menentukan biaya setiap penjualan dengan menghitung harga dari masing-masing semua unit yang dijual.

% 		\textbf{Penjualan tanggal 11 Mei} = 25 $\times$ 25 = \textbf{\${625}}

% 		\textbf{Penjualan tanggal 15 Mei} = 30 $\times$ 10 = \textbf{\${300}}

% 		\textbf{Penjualan tanggal 25 Mei} = 35 $\times$ 30 = \textbf{\${1,050}}

% 		\item Menjumlahkan semua biaya setiap penjualan
		
% 		Menggabungkan semua biaya setiap penjualan dari unit yang dijual.

% 		\textbf{Harga pokok barang yang tersedia = 625 + 300 + 1,050}

% 		\textbf{Harga pokok barang yang tersedia = \${1,975}}

% 		\item Temukan unit yang tersedia untuk dijual
		
% 		Totalkan semua unit yang tersedia untuk dijual dengan inventaris awal.

% 		\textbf{Unit yang tersedia untuk dijual = 30 + 50 + 15 + 25}

% 		\textbf{Unit yang tersedia untuk dijual = 120 unit}

% 		\item Gunakan rumus rata-rata tertimbang
		
% 		\textbf{WAC per unit = harga pokok barang yang tersedia / unit yang tersedia}
		
% 		\textbf{WAC per unit = 1,975 / 120}
		
% 		\textbf{WAC per unit = \${16.49}}

% 		Dari perhitungan tersebut, didapat biaya rata-rata tertimbang atau WAC-nya adalah \${16.49}
% 	\end{enumerate}

\section{EOQ}

\textit{Economic order quantity} (EOQ) merupakan jumlah persediaan yang digunakan untuk meminimalkan jumlah dan biaya pemesanan yang terkait dengan bahan baku atau persediaan barang dagangan. Intinya, EOQ merupakan \textit{set point} yang dibuat dan digunakan untuk menjadi acuan dalam membantu perusahaan meminimalkan total biaya persediaan.

Dua faktor penting yang menjadi penentu dalam menentukan \textit{economic order quantity} (EOQ) adalah biaya pemesanan dan biaya penyimpanan.

\begin{enumerate}
	\item Biaya pemesanan
	
	Biaya pemesanan merupakan biaya yang dikeluarkan setiap pesanan. Contoh hal yang termasuk biaya pemesanan adalah biaya pengiriman, biaya pemrosesan pembayaran, dan lain-lain.

	\item Biaya persediaan
	
	Biaya persediaan merupakan biaya yang dikeluarkan untuk menyimpan persediaan di toko atau gudang. Contoh hal yang termasuk dalam biaya penyimpanan adalah biaya sewa ruang penyimpanan, pajak properti, dan lain-lain.
\end{enumerate}

Formula atau rumus yang digunakan untuk menentukan EOQ adalah:

\begin{equation}
    \begin{split}
		EOQ
		&= \sqrt{\frac{2 \times D \times Co}{Ch}}
    \end{split}
\end{equation}

\begin{itemize}
	\item D = \textit{Demand per year} (Kebutuhan per tahun)
	\item Co = \textit{Cost per order} (Biaya per pesanan)
	\item Ch = \textit{Cost of holding per unit of inventory} (Biaya persediaan per unit)
\end{itemize}

\textbf{Contoh Kasus}

Sebuah material DX digunakan rutin setiap tahunnya. Data kebutuhan per tahun, biaya pemesanan, dan biaya persediaan per unit adalah sebagai berikut.

\begin{itemize}
	\item Kebutuhan tahunan = 2,400 unit
	\item Biaya per pesanan = \${10} per pesanan
	\item Biaya persediaan per unit = \${0.30} per unit
\end{itemize}

Diketahui:
\begin{itemize}
	\item D = \textit{Demand per year} (Kebutuhan per tahun) -> 2,400
	\item Co = \textit{Cost per order} (Biaya per pesanan) -> \${10}
	\item Ch = \textit{Cost of holding per unit of inventory} (Biaya persediaan per unit) -> \${0.30}
\end{itemize}

Maka, EOQ-nya adalah sebagai berikut.

\begin{equation}
    \begin{split}
		EOQ
		&= \sqrt{\frac{2 \times D \times Co}{Ch}} \\
		&= \sqrt{\frac{2 \times 2,400 \times 10}{0.30}} \\
		&= \sqrt{\frac{48,000}{0.30}} \\
		&= \sqrt{160,000} \\
		&= 400
    \end{split}
\end{equation}

Dapat dilihat bahwa EOQ dari material DX adalah sebesar 400 unit. Sekarang dapat dihitung berapa jumlah penjualan tahunan, biaya pemesanan tahunan, biaya penyimpanan tahunan, dan juga kombinasi dari biaya pemesanan tahunan dan biaya persediaan tahunan sebagai berikut.

\textbf{Jumlah penjualan tahunan}

= Kebutuhan tahunan / EOQ

= 2,400 unit / 400 unit

= 6 pesanan per tahun

\textbf{Biaya pemesanan tahunan}

= Jumlah penjualan tahunan * Biaya pemesanan per unit

= 6 pesanan * \${10}

= \${60}

\textbf{Biaya penyimpanan tahunan}

= Rata-rata unit * Biaya penyimpanan

= (400/2) * 0.3

= \${60}

\textbf{Kombinasi antara biaya pemesanan dan biaya penyimpanan}

= Biaya pemesanan tahunan + biaya penyimpanan tahunan

= \${60} + \${60}

= \${120}

\begin{table}[h!]
	\caption{Tabel Hasil Perhitungan EOQ}
	\label{tab:table4}
	\begin{tabular}{|m{0.1\linewidth}|m{0.1\linewidth}|m{0.13\linewidth}|m{0.13\linewidth}|m{0.16\linewidth}|m{0.13\linewidth}|} % <-- Alignments: 1st column left, 2nd middle and 3rd right, with vertical lines in between
	\hline
	\multicolumn{1}{|c|}{Jumlah Pesanan} & \multicolumn{1}{|c|}{Nilai} & \multicolumn{1}{|c|}{Rata-rata barang} & \multicolumn{3}{c|}{Biaya Pemesanan dan Penyimpanan} \\
	\cline{4-6}
	\multicolumn{1}{|c|}{Per Tahun} & \multicolumn{1}{|c|}{EOQ} & \multicolumn{1}{|c|}{dalam Persediaan} & Biaya Pemesanan & Biaya Penyimpanan & Biaya Gabungan \\
	\hline
		1 & 2,400 & 1,200 & 10 & 360 & 370\\ \hline
		2 & 1,200 & 600 & 20 & 180 & 200\\ \hline
		3 & 800 & 400 & 30 & 120 & 150\\ \hline
		4 & 600 & 300 & 40 & 90 & 130\\ \hline
		5 & 480 & 240 & 50 & 72 & 122\\ \hline
		\textbf{6} & \textbf{400} & \textbf{200} & \textbf{60} & \textbf{60} & \textbf{120}\\ \hline
		7 & 343 & 172 & 70 & 52 & 122\\ \hline
		8 & 300 & 150 & 80 & 45 & 125\\ \hline
	\end{tabular}
\end{table}


Pada Tabel 2.4, dapat dilihat bahwa dengan data yang sama menghasilkan hitungan yang berbeda tergantung dari berapa banyak jumlah penjualan tahunannya.

Dari hitungan EOQ yang sudah dilakukan sebelumnya, jumlah penjualan tahunan sebesar 6 pesanan per tahun mendapatkan biaya kombinasi yang lebih sedikit dan stabil dibandingkan dengan kurang atau lebih dari 6 pesanan per tahunnya. Hal ini dikarenakan jika semakin kecil angka penjualan tahunannya maka hal tersebut akan berdampak pada tingginya biaya penyimpanan yang menyebabkan ketidakseimbangan antara biaya pemesanan dan biaya penyimpanan. Sementara itu, jika penjualan pertahunnya itu tinggi maka hal tersebut akan berdampak pada tingginya biaya pemesanan yang menyebabkan hal yang serupa. Jadi, dapat disimpulkan bahwa perhitungan EOQ ini bersifat konstan terhadap biaya pemesanan dan biaya penyimpanan.

Dari kesimpulan diatas, ada beberapa hal yang harus diperhatikan dalam penggunaan metode EOQ ini, yaitu:

\begin{enumerate}
	\item Jumlah kebutuhan barang per periode stabil
	\item Hanya ada dua macam biaya yang relevan, yaitu biaya pemesanan dan biaya penyimpanan
	\item Biaya pemesanan selalu sama
	\item Biaya penyimpanan per unit selalu sama
	\item Usia barang tidak cepat rusak
	\item Harga barang tetap
	\item Barang tersedia tak terbatas.
\end{enumerate}

\section{ROP}

Pada buku Pengendalian Persediaan \citep{pengendalianpersediaan}, dijelaskan bahwa dalam EOQ ditentukan titik pemesanan kembali atau \textit{reorder point} yang biasa dikenal sebagai (ROP), yaitu jumlah persediaan tetap setiap kali pemesanan. ROP dilakukan bila persediaan bisa memenuhi kebutuhan produksi selama masa tenggang waktu pemesanan. ROP menghendaki pengecekan kartu catatan secara teratur.

Untuk menentukan waktu pemesanan kembali atau \textit{reorder point} dapat dilakukan dengan rumus berikut.

\begin{equation}
    \begin{split}
		ROP
		&= LT \times AU + SS
    \end{split}
\end{equation}

\begin{itemize}
	\item ROP = \textit{Reorder point}, yaitu tingkat dimana perusahaan harus memesan kembali. 
	\item LT = \textit{Leadtime}, yaitu masa kadaluarsa antara pemesanan sampai dengan kedatangan bahan.
	\item AU = \textit{Average usage}, yaitu pemakaian rata-rata dalam pemakaian tertentu.
	\item SS = \textit{Safety stok} yaitu besarnya persediaan atau bisa dibilang \textit{minimum inventory point}.
\end{itemize}

Dalam menutupi kebutuhan persediaan, hal yang perlu dilakukan adalah pemesanan bahan. Pemesanan bahan yang diperlukan pada saat persediaan mencapai titik tertentu (\textit{order point system}) dan pemesanan yang diperlukan pada saat waktu tertentu yang sudah ditetapkan telah tercapai (\textit{order cycle system}).

\subsection{Order Point System} 
	

	\textit{Order point system} adalah suatu sistem dimana pesanan dilakukan apabila persediaan yang ada telah mencapai tingkat tertentu. Jadi dengan sistem ini, ditentukan jumlah persediaan pada tingkatan tertentu yang merupakan batas tenggat waktu dilakukannya pemesanan yang disebut \textit{reorder point}. Dalam sistem ini, pesanan yang jumlahnya tetap dari bahan-bahan yang dipesan disebut dengan \textit{fixed order quantity system}. 

	Keuntungan dari sistem ini adalah pemantauan jumlah dan waktu pemesanan dapat dilakukan dengan mudah dan cepat.

	Dalam pelaksanaan sistem ini, dapat dilakukan dua variasi sebagai berikut.

	\begin{enumerate}
		\item \textit{Two bin and bag account system}
		
		Dengan cara ini, dapat digunakan dua kantong atau \textit{bin} dimana kantong pertama merupakan tempat persediaan bahan yang jumlahnya sama dengan jumlah persediaan pada tingkat \textit{order point} dan berfungsi sebagai persediaan cadangan. Sedangkan persediaan bahan-bahan selebihnya ditempatkan pada kantong kedua.

		Penggunaan bahan-bahan dimulai dari kantong kedua sampai habis dan ketika kantong kedua sudah habis maka diharuskan untuk melakukan pemesanan kembali.

		Sistem ini adalah sistem yang sederhana dan mudah untuk dilakukan pengendalian bahan ataupun pencatatan.

		\item \textit{One storage bin system}
		
		Dengan cara ini, hanya menggunakan satu kantong persediaan. Didalam kantong persediaan ini diadakan pembagian persediaan menjadi dua bagian. Bagian pertama dibagi untuk memenuhi kebutuhan bahan-bahan sehari-hari, sementara bagian kedua digunakan untuk memenuhi kebutuhan bahan-bahan selama periode pengisian kembali.

		Cara ini memberikan keuntungan berupa kesederhanaan dalam pencatatan persediaan.
	\end{enumerate}

\subsection{Order Cycle System}
	
	\textit{Order cycle system} adalah sistem pemesanan bahan yang dimana jarak antara pemesanan tetap, sebagai contoh tiap minggu atau tiap bulan. Jadi, dengan sistem ini ditentukan waktu pemesanan dengan jarak yang konstan. Karena didasarkan pada jarak waktu yang konstan, maka pemesanan dilakukan tanpa memperhatikan jumlah persediaan yang masih ada. \textit{Order cycle system} dapat digunakan untuk memantau persediaan barang yang mempunyai banyak jenis.

	Sistem ini termasuk salah satu sistem yang kaku dan mahal, karena setiap interval barang harus diperhatikan dan harus diperkirakan dahulu mengenai pemakaian barang tersebut di masa yang akan datang. Jika terdapat kesalahan perkiraan, maka dapat terjadi ketidakakuratan persediaan sehingga persediaan dapat berlebihan atau kehabisan persediaan.


% \textbf{Contoh Kasus}

% Ada seorang penjual baju gamis asal luar negeri. Dimisalkan pemasok tidak pernah mengalami kendala terkait stok barang. Akan tetapi, untuk pengambilan barang tersebut dibutuhkan waktu 5 hari.

% Setelah itu, pesanan penjual akan dikirim ke Indonesia dan memerlukan waktu paling cepat 14 hari.

% Sesampainya di Indonesia, barang tersebut harus diperiksa lagi oleh bea cukai dan memerlukan waktu selama satu minggu. Kemudian, barang tersebut baru bisa diterima melalui jalur darat selama 3 hari.

% Berdasarkan contoh kasus diatas, maka total \textit{lead time} adalah.

% \textbf{LT} = 5 + 14 + 7 + 3 = 29 Hari

% Data tersebut menjelaskan bahwa penjual harus mempunyai stok baju gamis selama masa tersebut untuk bisa dijual sampai pengiriman selanjutnya. Agar dapat terhindar dari kehabisan persediaan sebelum barang yang penjual pesan dari pemasok tiba, maka dapat dilakukan antisipasi \textit{demand}.

% Cara menghitung \textit{demand} adalah dengan mengalikan \textit{lead time} dengan nilai rata-rata penjualan harian si penjual.

% Misalkan penjual dapat menjual baju gamis sebanyak 15 gamis perhari, maka \textbf{\textit{Lead Time Demand}}-nya adalah.

% Lead Time Demand = Lead Time $\times$ Rata-rata Penjualan Perhari

% Lead Time Demand = 29 $\times$ 15 = 435.

% Artinya, penjual harus menyediakan 435 baju gamis untuk mengantisipasi pesanan pelanggan sehingga barang yang dikirim oleh pemasok tiba.

\section{Safety Stock}

Persediaan pengamanan atau bisa disebut sebagai  \textit{safety stock} adalah persediaan yang dilakukan untuk mengantisipasi ketidakpastian permintaan dan penyediaan. Apabila \textit{safety stock} tidak mampu mengantisipasi ketidakpastian tersebut, maka dapat terjadi kekurangan persediaan (\textit{stockout}).

Dalam menentukan \textit{safety stock}, dapat dilakukan dengan rumus berikut.

\begin{equation}
    \begin{split}
		SS
		&= (M - A) \times T 
    \end{split}
\end{equation}

\begin{itemize}
	\item SS = Safety stock
	\item M = Pemakaian maksimum per bulan
	\item A = Pemakaian rata-rata per bulan
	\item T = Waktu tunggu
\end{itemize}

Dengan adanya \textit{safety stock} akan mengantisipasi jika terjadi sesuatu yang menghambat pembelian sehingga stok barang persediaan masih ada untuk beberapa waktu kedepan.

\section{Pengendalian Persediaan}

Pengendalian Persediaan adalah suatu model yang digunakan untuk menyelesaikan masalah yang terkait dengan usaha pengendalian barang dalam suatu aktifitas perusahaan.

Persediaan yang terlalu berlebihan akan merugikan, karena berarti akan lebih banyak modal yang diperlukan, serta biaya yang diperlukan untuk persediaan.

Menurut Sunyoto (2012:225), Sistem pengendalian persediaan merupakan serangkaian pengendalian untuk menentukan tingkat persediaan yang harus dijaga. Sistem ini menentukan dan menjamin tersedianya persediaan yang tepat dalam kualitas dan waktu yang tepat. Jika persediaan terlalu sedikit dapat mengakibatkan resiko terjadinya kekurangan persediaan atau bisa dibilang \textit{stockout}. Bila persediaan dilebihkan, maka biaya penyimpanan dan modal yang diperlukan akan bertambah. Sebaliknya, jika persediaan dikurangi maka akan mengalami \textit{stockout} (kehabisan barang).
 
Menurut Assauri (2004), Pengendalian persediaan dapat dikatakan sebagai suatu kegiatan untuk menentukan tingkat dan komposisi dari persediaan sehingga perusahaan dapat melindungi kelancaran produksi dan penjualan serta kebutuhan-kebutuhan perusahaan dengan efisien.

Pada dasarnya, pengendalian persediaan akan mempermudah operasi perusahaan untuk memproduksi barang-barang, disimpan di gudang dan sampai ke konsumen. Persediaan yang terlalu besar (\textit{overstock}) merupakan pemborosan karena menyebabkan tingginya beban biaya untuk inventaris barang-barang tersebut, sementara jika persediaan terlalu kecil maka dapat menyebabkan proses produksi terhenti sehingga konsumen akan pergi karena permintaannya tidak terpenuhi. Intinya, pengendalian persediaan akan mempermudah atau memperlancar jalannya operasi perusahaan dalam mengelola barang. 

Dalam pengendalian persediaan terdapat tiga aspek yang perlu dipertimbangkan, yaitu:

\begin{enumerate}
	\item Sistem pengadaan persediaan
	
	Perusahaan harus menentukan sistem pengadaan persediaan dengan memperhatikan faktor-faktor yang mempengaruhi pengendalian persediaan.

	\item Penentuan jumlah persediaan
	
	Penentuan jumlah persediaan merupakan aspek penting dalam pengendalian persediaan, kekurangan dan kelebihan jumlah persediaan akan mempengaruhi tingkat keuntungan yang diperoleh perusahaan.

	\item Administrasi persediaan
	
	Dalam menjalankan pengendalian persediaan, diperlukan administrasi persediaan yang baik dan teratur.
\end{enumerate}

Agar pengendalian persediaan dapat dilakukan dengan maksimal, menurut Assauri (2004:176) ada faktor-faktor yang harus dipertimbangkan dalam menjalankan pengendalian persediaan, antara lain:

\begin{enumerate}
	\item Adanya fasilitas pergudangan yang cukup luas dan teratur
	\item Adanya sistem administrasi pencatatan dan pemeriksaan atas penerimaan dan pengeluaran barang
	\item Sumber daya yang menguasai sistem administrasi pengendalian persediaan yang digunakan perusahaan
	\item Perencanaan untuk mengganti barang yang telah digunakan dan barang yang sudah lama berada dalam gudang sehingga usang
	\item Informasi dari bagian produksi tentang sifat teknis barang, daya tahan produk dan lamanya produksi, untuk melakukan perencanaan pengendalian persediaan
	\item Informasi dari bagian penjualan tentang tingkat penjualan produk perusahaan, sehingga bagian persediaan bisa menentukan besarnya persediaan yang seharusnya ada sehingga tidak terjadi kekurangan persediaan yang mengakibatkan pesanan konsumen tidak terpenuhi.
\end{enumerate}

\section{Penentuan Harga Transfer}

Pada buku \textit{Management Control} \citep{manajemencontrol}, dijelaskan bahwa penentuan harga transfer atau \textit{transfer pricing} merupakan proses harga penentuan harga yang ditetapkan dalam transaksi penjualan dan pembelian diantara berbagai unit organisasi pada kelompok perusahaan atau instansi yang sama.

% Dalam menentukan harga transfer, prinsip dasarnya adalah bahwa harga transfer sebaiknya serupa dengan harga yang akan dikenakan seandainya produk tersebut dijual ke konsumen luar atau dibeli dari pemasok luar. Ketika suatu perusahaan membeli atau menjual produk
\subsection{Karakteristik Harga Transfer}
\subsection{Syarat Terpenuhinya Harga Transfer}
\subsection{Tujuan Penentuan Harga Transfer}
\subsection{Dampak Penentuan Harga Transfer}
\subsection{Kebijakan Penentuan Harga Transfer}
\subsection{Prinsip Dasar Penentuan Harga Transfer}
\subsection{Metode Penentuan Harga Transfer}
\subsection{Administrasi Harga Transfer}
\subsection{Harga Transfer Divisi Terintegrasi}